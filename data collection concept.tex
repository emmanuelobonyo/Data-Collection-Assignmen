% Generated by GrindEQ Word-to-LaTeX 
\documentclass{article} %%% use \documentstyle for old LaTeX compilers

\usepackage[english]{babel} %%% 'french', 'german', 'spanish', 'danish', etc.
\usepackage{amssymb}
\usepackage{amsmath}
\usepackage{txfonts}
\usepackage{mathdots}
\usepackage[classicReIm]{kpfonts}
\usepackage[dvips]{graphicx} %%% use 'pdftex' instead of 'dvips' for PDF output

% You can include more LaTeX packages here 


\begin{document}

%\selectlanguage{english} %%% remove comment delimiter ('%') and select language if required


\noindent \textbf{Background to the Problem}

\noindent Makerere University has more than ${}^{[1] }$forty thousand students; all of which are pursuing different levels of education, from certificate level to ${}^{[2]}$ Doctor of Philosophy (PhD) Level. Compared with the much less\textit{ }number of teaching staff (lecturers) at the university -- one lecturer at times may be responsible for upto 500 students at a particular time) This a ratio of 1:400. This is absurd and obviously deems meaningful interaction between students and lecturers impossible. At times students have striked because apparently lecturers are teaching them. I believe if there was a medium that enabled safe and honest communication this problem would be almost nonexistent. 

\noindent Many times, my claasmates and I have gone to see certain lecturers only to find that they are very busy tending to other important matters or, even, unavailable in their offices or respective places of work for various reasons. This sometimes happens over and over again until one of the parties involved ends up giving up because they feel like valuable time is being wasted. This has meant that sometimes students have to rely on \textit{luck }to finally meet a lecturer -- that is if they don't meet during class time, which is not unusual seeing as due to the large number of students, some students easlily miss lecturers unnoticed, and sometimes the lecturers don't turn up for lectures (At times students have striked because apparently lecturers are teaching them. I believe if there was a medium that enabled safe and honest communication this problem would be almost nonexistent.). Often times communication is sent through third parties, for example class representatives -- and by the time the intended communication reaches the intended audience, it's either been slightly altered, miscommunicated or a tinge late. This, obviously, can create disastrous problems -- usually for the students -- depending on the urgency and complicacy of the information. 

\noindent The Major Objective of this research is to find out how many teaching institutions in Uganda are affected by this problem and to eventually come up with a sustainable solution to this problem. 

\noindent The Research process will start on 1${}^{st}$ March 2018 and end on 30${}^{th}$ November 2018. We believe this is enough time for us to cut across the country looking for solutions. 

\noindent Questions to be asked during the data collection process. 

\begin{enumerate}
\item  Name:

\item  Gender:

\item  Age:

\item  Nationality:

\item  Profession:

\item  Lecturer / Teacher

\item  Student

\item  University/ Tertiary institution/School

\item  Level of Interaction with Lecturer / Teacher and/or Student

\item  Very Good

\item  Good

\item  Average

\item  Poor

\item  Very Poor

\item  Do you think the level of interaction can be improved? 

\item  Yes

\item  No

\item  Maybe

\item  If Yes/Maybe, kindly suggest any ways this cvan be achieved.

\item  
\end{enumerate}

\noindent 


\end{document}

